\documentclass[10pt]{article}
\usepackage[a4paper, left=1in, right=1in, top=1.5in, bottom=1.5in]{geometry}
\usepackage[colorlinks=true, linkcolor=blue]{hyperref}
\usepackage{amsmath} % for math environments
\usepackage{raynotes}  % Load the custom package
\usepackage{xcolor}
\usepackage{framed}
\usepackage{tikz-cd}
\usepackage{tikz} % for creating graphics and annotations
\usetikzlibrary{tikzmark, positioning} % for marking nodes
\usepackage{wrapfig} 

% Customize leftbar
\definecolor{leftbarcolor}{gray}{0.75}  % Define color (optional)
\renewenvironment{leftbar}{%
    \def\FrameCommand{\vrule width 1pt \hspace{10pt}}%  Adjust width (1pt) and spacing (10pt)
    \MakeFramed{\advance\hsize-\width\FrameRestore}}%
 {\endMakeFramed}

\title{Notes on Set Theory}
\author{Ray Li}
\date{\today}

\begin{document}

\maketitle

\tableofcontents
\newpage\
\section{Chapter 6: Cardinal Numbers and The Axim of Choice}

\subsection{Material Notes}
\subsubsection{FINITE SETS}
In the Book Page 134 to 135, while proving the case 1, the book mentioned

% \textbf{Definitions:}
\begin{leftbar}
    \textbf{Pigeonhole Principle:} No natural number is equinumerous to a proper subset of itself.

    \textbf{Proof} Assume that \(f\) is a one-to-one function from the set \(n\) into the set \(n\). We will show that \(\text{ran} f\) is all of the set \(n\) (and not a proper subset of \(n\)). This suffices to prove the theorem.
    We use induction on \(n\). Define:
    \[
    T = \{ n \in \omega \mid \text{any one-to-one function from } n \text{ into } n \text{ has range } n \}.
    \]
    Then \(0 \in T\); the only function from the set \(0\) into the set \(0\) is \(\emptyset\) and its range is the set \(0\). Suppose that \(k \in T\) and that \(f\) is a one-to-one function from the set \(k^+\) into the set \(k^+\). We must show that the range of \(f\) is all of the set \(k^+\); this will imply that \(k^+ \in T\). Note that the restriction \(f \upharpoonright k\) of \(f\) to the set \(k\) maps the set \(k\) one-to-one into the set \(k^+\).

    \textbf{Case 1} Possibly the set \(k\) is closed under \(f\). Then \(f \upharpoonright k\) maps the set \(k\) into the set \(k\). Then because \(k \in T\) we may conclude that \(\text{ran} (f \upharpoonright k)\) is all of the set \(k\). Since \(f\) is one-to-one, the only possible value for \(f(k)\) is the number \(k\). Hence \(\text{ran} f\) is \(k \cup \{k\}\), which is the set \(k^+\).  
\end{leftbar}
\raynote{
    Here the \textbf{Case 1} should have more explanation: \\
    We know that \(k\) is closed under \(f\) and \(\text{ran} (f \upharpoonright k) = k\). Then why do we have \(\text{ran} f = k \cup \{k\}\)? This is because of the following argument: \\
    \(f\) is one-to-one. We also know that \(k \notin k\) (otherwise we would form Russell's paradox). The preimage \(f^{-1}[\{f(k)\}]\) (the preimage of \(f(k)\) under \(f\)) can only contain one element since \(f\) is one-to-one, and \(k \in f^{-1}[\{f(k)\}]\) because the preimage of \(f(k)\) must contain \(k\). Thus, \(\text{ran} f = \text{ran}(f \upharpoonright k) \cup \text{ran}(f \upharpoonright \{k\}) = k \cup \{k\}\).
    }
\subsubsection{CARDINAL ARITHMETIC}

In Page 139 to 140, while proving Theorem 6H, the book mentioned that
\begin{leftbar}
    \textbf{Theorem 6H} Assume that $K_1 \approx K_2$ and $L_1 \approx L_2$.

    \begin{enumerate}
        \item[(a)] If $K_1 \cap L_1 = K_2 \cap L_2 = \emptyset$, then $K_1 \cup L_1 \approx K_2 \cup L_2$.
        \item[(b)] $K_1 \times L_1 \approx K_2 \times L_2$.
        \item[(c)] ${}^{(L_1)}K_1 \approx {}^{(L_2)}K_2$.
    \end{enumerate}

\end{leftbar}
\raynote{Theorem 6H: More perspectives}

We may also prove that \(H\) is a bijection by using the theorem that a function is a bijection if and only if it has an inverse (i.e., its left and right inverses coincide). Thus, the remaining task is to find the inverse. This is straightforward to do. Since \(H(j) = f \circ j \circ g^{-1}\), we can express \(j\) as \(j = f^{-1} \circ H(j) \circ g\), which implies that \(H^{-1}(i) = f^{-1} \circ i \circ g\). It is easy to verify that such an \(H^{-1}\) is indeed the inverse by composing \(H\) with \(H^{-1}\) both on the left and the right.

Another approach to prove that \(H\) is a bijection between two function spaces is by demonstrating that \(H\) is both injective and surjective:
\begin{itemize}
    \item \textbf{\(H\) is surjective:} For every \(i \in {}^{(L_2)}K_2\), there exists a \(j \in {}^{(L_1)}K_1\) such that \(j = f^{-1} \circ i \circ g\). It is straightforward to verify that \(H(j) = i\).
    \item \textbf{\(H\) is injective:} We need to show that if \(H(j) = i\), then \(j\) must have the form \(j = f^{-1} \circ i \circ g\). In other words, \(H(j) = i\) if and only if \(j = f^{-1} \circ i \circ g\). This is not difficult to verify: \(H(f^{-1} \circ i \circ g) = i\). Additionally, since \(H(j) = f \circ j \circ g^{-1} = i\), it follows that \(j = f^{-1} \circ i \circ g\). After establishing this, if \(H\) maps \(j_1\) and \(j_2\) to the same element \(i\), then we must have \(j_1 = f^{-1} \circ i \circ g\) and \(j_2 = f^{-1} \circ i \circ g\). Furthermore, since function composition is a function, we conclude that \(j_1 = j_2\).
\end{itemize}

In both of the above methods, we do not need to explicitly unfold what \(i\) and \(j\) are; we treat them as objects themselves rather than focusing on their relations. This, in a certain sense, makes the argument more abstract and simpler. However, in both approaches, we need to use the fact that function composition is itself a function, meaning that \(K_g: f \mapsto f \circ g\) is a function of \(f\), and similarly, \(K'_g: f \mapsto g \circ f\) is also a function of \(f\).

In Page 141 , the book mentioned that

\begin{leftbar}
    \textbf{5.} Recall that \(\emptyset^K = \{\emptyset\}\) for any set \(K\) and that \(K^{\emptyset} = \emptyset\) for nonempty \(K\). In terms of cardinal numbers, these facts become
    \[
    \kappa^0 = 1 \quad \text{for any } \kappa,
    \]
    \[
    0^{\kappa} = 0 \quad \text{for any nonzero } \kappa.
    \]

    In particular, \(0^0 = 1\).
\end{leftbar}
\raynote{More on cardinal arithmetic} \\

\textbf{1. Understanding \({}^\emptyset K = \{\emptyset\}\)} \\ 

\textbf{Notation:} \({}^A B\) represents the set of all functions from set \(A\) to set \(B\).

\textbf{Case:} When \(A = \emptyset\) (the empty set) and \(B = K\) (any set).

\textbf{Explanation:}

\textbf{Definition of a Function:} A function \(f: A \to B\) is a set of ordered pairs \((a, f(a))\) where each \(a \in A\) is paired with exactly one \(f(a) \in B\). \\

\textbf{When \(A = \emptyset\):} \\ 
There are \textbf{no elements} in \(A\) to pair with elements in \(B\). \\ 
Therefore, the \textbf{only possible function} is the \textbf{empty function}, which is the empty set \(\emptyset\). \\
\textbf{Conclusion:} Since there is exactly \textbf{one} function from \(\emptyset\) to \(K\), we have:
  \[
  {}^\emptyset K = \{\emptyset\}
  \]
\textbf{Cardinal Arithmetic Interpretation:}
    \begin{itemize}
        \item The number of such functions is \textbf{1}.
        \item Hence, for any cardinal \(\kappa\):
        \[
        \kappa^0 = 1
        \]
    \end{itemize}

\textbf{2. Understanding \({}^K \emptyset = \emptyset\) for Nonempty \(K\)}

\textbf{Notation:} \({}^K \emptyset\) represents the set of all functions from \(K\) to the empty set \(\emptyset\).

\textbf{Case:} When \(K\) is \textbf{nonempty} and the codomain is \(\emptyset\).

\textbf{Explanation:}

\textbf{Definition of a Function:} A function \(f: K \to \emptyset\) must assign to \textbf{every} element \(k \in K\) an element \(f(k) \in \emptyset\). \\
\textbf{Problem:} The empty set \(\emptyset\) has \textbf{no elements}. Therefore, there is \textbf{no possible way} to assign a value \(f(k)\) for any \(k \in K\). \\
\textbf{Conclusion:} Since it is \textbf{impossible} to define such a function when \(K\) is nonempty, there are \textbf{no functions} from \(K\) to \(\emptyset\):
  \[
  {}^K \emptyset = \emptyset
  \] \\ 
\textbf{Cardinal Arithmetic Interpretation:}
    - The number of such functions is \textbf{0}.
    - Hence, for any nonzero cardinal \(\kappa\):
      \[
      0^\kappa = 0
      \]
\subsubsection{ORDERING CARDINAL NUMBERS}
Page 146, in part
\begin{leftbar}
    \textbf{Examples 1.} 
    \begin{enumerate}
        \item If \(A \subseteq B\), then \(\text{card } A \leq \text{card } B\). Conversely, whenever \(\kappa \leq \lambda\), there exist sets \(K \subseteq L\) with \(\text{card } K = \kappa\) and \(\text{card } L = \lambda\). To prove this, start with any sets \(C\) and \(L\) of cardinality \(\kappa\) and \(\lambda\), respectively. Then \(C \subseteq L\), so there is a one-to-one function \(f\) from \(C\) into \(L\). Let \(K = \text{ran } f\); then \(C \approx K \subseteq L\).
        \item For any cardinal \(\kappa\), we have \(0 \leq \kappa\).
        \item For any finite cardinal \(n\), we have \(n < \aleph_0\). (Why?) For any two finite cardinals \(m\) and \(n\), we have:
        \[
        m \in n \implies m \subseteq n \implies m \leq n.
        \]
    \end{enumerate}
\end{leftbar}
About the `(Why?)` part. This is because any natural number \(n\) is a proper subset of \(\omega\). Thus the embedding map is the injection. Furthermore there is no surjection between \(n\) to \(\omega\) since \(n^+ = n \cap \{n\} \notin n\). Thus \(n \subsetneq \omega \& n \neq \omega\).

\subsubsection{AXIOM OF CHOICE}
In Page 156, for the Theorem 6N:
\begin{leftbar}
    \begin{itemize}
        \item[(a)] For any infinite set \( A \), we have \( \omega \leq A \).
        \item[(b)] \( \aleph_0 \leq \kappa \) for any infinite cardinal \( \kappa \).
    \end{itemize}
\end{leftbar}
\raynote{Intuitive idea on Theorem 6N}

\begin{tikzcd}
    B + \omega \arrow[r]  \arrow[d, "bijection"'] & B + (\omega - C) \arrow [l, "bijection"] \arrow[d] \\
    A  \arrow[u] \arrow[r, "?"'] & A - C \arrow[u, "bijection"']
\end{tikzcd}

where \(C \neq \emptyset \).
Clearly since composition of bijections are bijection, \(A\) and \(A - C\) has a bijection.

\subsubsection{COUNTABLE SETS}
Page 160:
\begin{leftbar}
    \textbf{Theorem 6Q} \textit{A countable union of countable sets is countable. That is, if \(\mathcal{A}\) is countable and if every member of \(\mathcal{A}\) is a countable set, then \(\bigcup \mathcal{A}\) is countable.}

    \textbf{Proof:} We may suppose that \(\emptyset \notin \mathcal{A}\), for otherwise we could simply remove it without affecting \(\bigcup \mathcal{A}\). We may further suppose that \(\mathcal{A} \neq \emptyset\), since \(\bigcup \emptyset\) is certainly countable. Thus, \(\mathcal{A}\) is a countable (but nonempty) set from \(\omega \times \omega\) onto \(\bigcup \mathcal{A}\). We already know of functions from \(\omega\) onto \(\omega \times \omega\), and the composition will map \(\omega\) onto \(\bigcup \mathcal{A}\), thereby showing that \(\bigcup \mathcal{A}\) is countable.

    Since \(\mathcal{A}\) is countable but nonempty, there is a function \(G\) from \(\omega\) onto \(\mathcal{A}\). Informally, we may write
    \[
    \mathcal{A} = \{G(0), G(1), \ldots\}.
    \]
    (Here \(G\) might not be one-to-one, so there may be repetitions in this enumeration.) We are given that each set \(G(m)\) is countable and nonempty.

    Hence for each \(m\) there is a function from \(\omega\) onto \(G(m)\). We must use the axiom of choice to select such a function for each \(m\).

    Because the axiom of choice is a recent addition to our repertoire, we will describe its use here in some detail. Let \(H: \omega \to \omega(\bigcup \mathcal{A})\) be defined by
    \[
    H(m) = \{g \mid g \text{ is a function from } \omega \text{ onto } G(m)\}.
    \]
    We know that \(H(m)\) is nonempty for each \(m\). Hence there is a function \(F\) with domain \(\omega\) such that for each \(m\), \(F(m)\) is a function from \(\omega\) onto \(G(m)\).

    To conclude the proof we have only to let \(f(m, n) = F(m)(n)\). Then \(f\) is a function from \(\omega \times \omega\) onto \(\bigcup \mathcal{A}\).

\end{leftbar}
\raynote{Intuation on countable set of countable sets is countable}

To gain a more intuitive understanding, observe the illustration: Since \(\mathcal{A}\) is countable, it has an enumeration given by \(\mathcal{A} = \{G(0), G(1), \ldots\} = \{G_0, G_1, \ldots\}\), where each \(G_m\) is another countable set. More explicitly, we can represent each set as follows:
\[
G_0 = \{G_0^0, G_0^1, G_0^2, G_0^3, \ldots, G_0^m, \ldots\}
\]
\[
G_1 = \{G_1^0, G_1^1, G_1^2, G_1^3, \ldots, G_1^m, \ldots\}
\]
\[
\vdots
\]
\[
G_n = \{G_n^0, G_n^1, G_n^2, G_n^3, \ldots, G_n^m, \ldots\}
\]
\[
\vdots
\]
Thus, this construction clearly forms an injection into \(\omega \times \omega\).


\subsection{Excercise Answers}
\begin{question}
    \textbf{E 6} Let $\kappa$ be a nonzero cardinal number. Show that there does not exist a set to which every set of cardinality $\kappa$ belongs.
\end{question}
\begin{proof}
    Proof by contradiction (BWOC): Assume that such a set exists. Let 
    \[ 
    A = \{ K \mid |K| = \kappa \} 
    \]
    be the set containing all sets with cardinality $\kappa$.

    Consider 
    \[
    \bigcup A,
    \]
    the union of all sets in $A$. By construction, $\bigcup A$ would be a set that contains every possible set of cardinality $\kappa$.

    However, this leads to a contradiction since such a set does not exist in the framework of standard set theory (e.g., due to limitations implied by Russell's paradox or cardinality constraints).

    Therefore, there does not exist a set $A$ such that it contains every set of cardinality $\kappa$.
\end{proof}

\begin{question}
    \textbf{E 7}
    Assume that $A$ is finite and $f: A \rightarrow A$. Show that $f$ is one-to-one if and only if $\text{ran } f = A$.
\end{question}
Given that $\text{ran } f = A$, this implies that $f$ is surjective.
Thus, the statement is equivalent to proving that: \textit{``Assume that $A$ is finite and $f: A \rightarrow A$. Show that $f$ is one-to-one if and only if $f$ is surjective.''}
\begin{proof}
Assume $A$ \text{ is finite.}
    \begin{itemize}
        \item (\(\Rightarrow\)) \textbf{(By way of contradiction):} Assume that $f$ is injective but not surjective. Then, the image of $f$ under $A$, denoted by $f[A]$, is a proper subset of $A$, meaning $f[A] \subsetneq A$. Thus, we have an injective function $f: A \rightarrow f[A]$. By construction, $f[A] = f[A]$, and therefore, $f$ is surjective onto $f[A]$. Hence, $f$ is a bijection from $A$ to $f[A]$, where $f[A] \subsetneq A$. According to Corollary 6D, this implies that $A$ is equinumerous to a proper subset of itself, which contradicts the assumption that $A$ is finite.

        \item (\(\Leftarrow\)) \textbf{(By way of contradiction):} Assume that $f$ is surjective but not injective. By the axiom of choice, there exists an injection, call it $g$, from $A$ to the preimage of $A$ under $f$, which is a proper subset of $A$. Such a function $g$ would then be a bijection from the preimage of $A$ to $A$. This again implies that $A$ is equinumerous to a proper subset of itself, leading to the conclusion that $A$ is infinite. This contradicts our original assumption that $A$ is finite.
    \end{itemize}
\end{proof}
\begin{question}
    \textbf{E 8}
    Prove that the union of two finite sets is finite (Corollary 6K), without any use of arithmetic.
\end{question}
\begin{proof}
    Let \(|A| \approx n\) and \(|B| \approx m\), where \(n, m \in \omega\). Assume that \(A\) and \(B\) are disjoint. By the definition of equinumerosity, there exist functions \(h: A \rightarrow n\) and \(g: B \rightarrow m\). We define a function \(f: A \cup B \rightarrow n + m\) (as defined earlier) such that

    \[
    f(x) = 
    \begin{cases} 
    h(x) & \text{if } x \in A \\
    n + g(x) & \text{if } x \in B
    \end{cases}
    \]

    This function is well-defined since \(A\) and \(B\) are disjoint. It is straightforward to verify that such an \(f\) is a bijection with an inverse \(f^{-1}: n + m \rightarrow A \cup B\) defined as follows:

    \[
    f^{-1}(k) = 
    \begin{cases} 
    h^{-1}(k) & \text{if } k \leq n \\
    g^{-1}(k - n) & \text{if } k > n
    \end{cases}
    \]

    If \(A\) and \(B\) are not disjoint, then let \(A \cap B = C \neq \emptyset\). Replace each element in \(C\) by elements not present in \(A \cup B\) to form a new set \(C'\). Let \(A' = (A - C) \cup C'\). We have \(|A'| = |A|\) and \(A \cup B \subseteq A' \cup B\), which forms a disjoint union. By Lemma 6F, there exists a \(k < n + m\) such that \(|A \cup B| = k\).

    Therefore, in both cases, \(A \cup B\) is finite.
\end{proof}
\begin{question}
    \textbf{E 9}
    Prove that the Cartesian product of two finite sets is finite (Corollary 6K), without any use of arithmetic.
\end{question}
\begin{proof}
    We will use induction to prove this statement:

    Let 
    \[
    S = \{ m \mid \forall n \in \omega, \forall A, B \text{ such that } |A| = m, |B| = n, |A \times B| = m \times n \},
    \]
    where \(\times\) is defined as the Cartesian product for natural numbers.

    \begin{enumerate}
        \item \textbf{Base Case:} \(0 \in S\). For any set \(B\), \(\emptyset \times B = \emptyset\). Thus, \(|\emptyset \times B| = 0 \times |B| = 0\) since \(B\) is finite and therefore \(|B| \in \omega\). Furthermore, \(1 \in S\) since for any singleton \(\{a\}\) and any set \(B\), there exists a bijection between \(\{a\} \times B\) and \(B\) given by the function \(f: \{a\} \times B \rightarrow B\) defined as \(f(a, x) = x\). The inverse function is \(g(x) = (a, x)\). Therefore, \(|\{a\} \times B| = 1 \times |B| = |B|\).
        
        \item \textbf{Inductive Step:} Assume \(k \in S\). We want to show that \(k^+ \in S\), where \(k^+\) denotes the successor of \(k\).
        
        To show \(k^+ \in S\), we need to prove that for all \(m \in \omega\) and for all sets \(A\) and \(B\) such that \(|A| = k^+\) and \(|B| = m\), we have \(|A \times B| = k^+ \times m\).
        
        Since we have already proven that \(0 \in S\), the induction starts with \(k \geq 1\), meaning \(k^+ \geq 2\).
        
        Let \(A\) be a set with \(|A| = k^+\). Since \(k^+ \geq 2\), there exists an element \(a \in A\). Let \(A' = A \setminus \{a\}\). Therefore, \(|A'| = k\), and by the induction hypothesis, \(|A' \times B| = k \times m\). Since \(\{a\}\) is a singleton, we have \(|\{a\} \times B| = 1 \times m = m\) by the induction hypothesis.
        
        Therefore,
        \[
        |A \times B| = |(A' \times B) \cup (\{a\} \times B)| = k \times m + m = k^+ \times m
        \]
        by the definition of multiplication for natural numbers.
    \end{enumerate}

    Hence, by induction, the Cartesian product of two finite sets is finite (since natural number is closed under multiplication).
\end{proof}
\begin{question}
    \textbf{E 10 to 12}
    For Excercises 10 to 12, proving all the Theorem 6I is suffices.
\end{question}
In Page 142
\begin{leftbar}
    \textbf{Theorem 6I} For any cardinal numbers \(\kappa\), \(\lambda\), and \(\mu\):

    \begin{enumerate}
        \item \(\kappa + \lambda = \lambda + \kappa\) and \(\kappa \cdot \lambda = \lambda \cdot \kappa\).
        \item \(\kappa + (\lambda + \mu) = (\kappa + \lambda) + \mu\) and \(\kappa \cdot (\lambda \cdot \mu) = (\kappa \cdot \lambda) \cdot \mu\).
        \item \(\kappa \cdot (\lambda + \mu) = \kappa \cdot \lambda + \kappa \cdot \mu\).
        \item \(\kappa^{\lambda + \mu} = \kappa^\lambda \cdot \kappa^\mu\).
        \item \((\kappa \cdot \lambda)^\mu = \kappa^\mu \cdot \lambda^\mu\).
        \item \((\kappa^\lambda)^\mu = \kappa^{\lambda \cdot \mu}\).
    \end{enumerate}

    \textbf{Proof} Take sets \(K\), \(L\), and \(M\) with \(\text{card } K = \kappa\), \(\text{card } L = \lambda\), and \(\text{card } M = \mu\); for convenience, choose them in such a way that any two are disjoint. Then each of the equations reduces to a corresponding statement about equinumerous sets. For example, \(\kappa \cdot \lambda = \text{card } (K \times L)\) and \(\lambda \cdot \kappa = \text{card } (L \times K)\); consequently, showing that \(\kappa \cdot \lambda = \lambda \cdot \kappa\) reduces to showing that \(K \times L \approx L \times K\). Listed in full, the statements to be verified are:

    \begin{enumerate}
        \item \(K \cup L \approx L \cup K\) and \(K \times L \approx L \times K\).
        \item \(K \cup (L \cup M) \approx (K \cup L) \cup M\) and \(K \times (L \times M) \approx (K \times L) \times M\).
        \item \(K \times (L \cup M) \approx (K \times L) \cup (K \times M)\).
        \item \({}^{(L \cup M)} K \approx {}^K L \times {}^k M\).
        \item \({}^M (K \times L) \approx {}^M K \times {}^M L\).
        \item \({}^{M} ({}^L K) \approx {}^{(L \times M)} K\).
    \end{enumerate}
\end{leftbar}
I will use the second part of 1 to 6 (on the set) to prove the theorem.
\begin{proof} For the simple ones I will simple list the fucntion without justifying that it is bijection.
    \begin{enumerate}
        \item \(K \cup L = L \cup K\) by algebra of set (Page 27). f: \(K \times L \rightarrow L \times K\) s.t. \(<k, l> \mapsto <l, k>\)
        \item \(K \cup (L \cup M) = (K \cup L) \cup M\) by algebra of set (Page 27). f: \(K \times (L \times M) \rightarrow (K \times L) \times M\) s.t. \(<k, <l, m>> \mapsto <<k, l>, m>\).
        \item \(K \times (L \cup M) = (K \times L) \cup (K \times M)\) by Excercise 2 Chapter3.
        \item \(H: {}^{(L \cup M)} K \rightarrow {}^K L \times {}^k M\) \(s.t. f \mapsto <f \upharpoonright {}_L, f \upharpoonright {}_{M - L}>\).
        \item \(H: {}^M (K \times L) \rightarrow {}^M K \times {}^M L\). Let \(f \in {}^M (K \times L) \), then \(\forall m \in M f: m \mapsto f(m) = <k, l> = <f_1(m), f_2(m)>\). Thus let \(H: f \mapsto <f_1, f_2>\).
        \item Proven in book. 
    \end{enumerate}
\end{proof}

\begin{question}
    \textbf{E 13}
    Show that a finite union of finite sets is finite. That is, show that if \(B\) is a finite set whose members are themselves finite sets, then \(\bigcup B\) is finite.
\end{question}
\begin{proof}
    In Chapter 6 E 3, we have proved that two finite set's union is finite. Do induction on natural number, we can prove that any finite union of finite sets is finite.
\end{proof}
\begin{question}
    \textbf{14}
    Define a \textit{permutation} of \(K\) to be any one-to-one function from \(K\) onto \(K\). We can then define the factorial operation on cardinal numbers by the equation
    \[
    \kappa! = \text{card} \{ f \mid f \text{ is a permutation of } K \},
    \]
    where \(K\) is any set of cardinality \(\kappa\). Show that \(\kappa!\) is well defined, i.e., the value of \(\kappa!\) is independent of just which set \(K\) is chosen.

\end{question}
\begin{proof}
    To show that \(\kappa!\) is well-defined, let \(\kappa\) be a cardinal number, and consider any two sets \(A\) and \(B\) such that \(|A| = |B| = \kappa\).

    Since \(A \approx A\), let \(f_A\) be a bijection \(f_A: A \rightarrow A\). Similarly, let \(f_B\) be a bijection \(f_B: B \rightarrow B\). Since \(A \approx B\), there exists a bijection \(f: A \rightarrow B\).

    We can establish a bijection between the set of permutations of \(A\) (denoted by \(\mathrm{perm}(A)\) or \(\mathrm{sym}(A)\)) and the set of permutations of \(B\) (denoted by \(\mathrm{perm}(B)\) or \(\mathrm{sym}(B)\)). This bijection can be constructed as follows: for any \(f_A \in \mathrm{perm}(A)\), we define a corresponding permutation of \(B\) by \(f_A' = f \circ f_B \circ f^{-1}\) (this is a conjugation by \(f\)).

    This construction is clearly a bijection between \(\mathrm{perm}(A)\) and \(\mathrm{perm}(B)\). Thus, the value of \(\kappa!\) is independent of the particular choice of the set \(K\) with cardinality \(\kappa\), proving that the factorial operation on cardinal numbers is well-defined.
\end{proof}
\begin{question}
    \textbf{E 15}
    Show that there is no set \(\mathcal{A}\) with the property that for every set there is some member of \(\mathcal{A}\) that dominates it.
\end{question}
\begin{proof}
    To prove that there is no such set \(\mathcal{A}\) with the stated property, we proceed by contradiction.

    \textbf{By way of contradiction (BWOC):} Assume that such a set \(\mathcal{A}\) exists. Let \(\mathcal{A}' = \bigcup \mathcal{A}\). By the definition of \(\mathcal{A}\), there exists a set \(K \in \mathcal{A}\) such that \(K\) dominates \(\mathcal{A}'\). Therefore, there is an injection from \(\mathcal{A}'\) to \(K\).

    By the definition of \(\mathcal{A}\), the power set \(2^K\) (the set of all subsets of \(K\)) is dominated by some element \(K' \in \mathcal{A}\). This means that there exists an injection from \(2^K\) to \(K'\). Since \(K' \in \mathcal{A}\) and \(\mathcal{A}' = \bigcup \mathcal{A}\), we have \(K' \subseteq \mathcal{A}'\). Therefore, the embedding of \(K'\) into \(\mathcal{A}'\) is a natural injection.

    Furthermore, we know that \(\mathcal{A}'\) injects into \(K\), and \(K\) injects into \(2^K\), since the power set of any set strictly dominates the set itself. Thus, we have an injection from \(2^K\) into \(\mathcal{A}'\). By the transitivity of injections (i.e., the property that injections are preserved under composition), this implies that \(2^K\) injects into \(K\), which contradicts the fact that the power set of a set always strictly dominates the set itself.

    Therefore, our initial assumption that such a set \(\mathcal{A}\) exists must be false.
\end{proof}

\begin{question}
    \textbf{E 16}
    Show that for any set \(S\) we have \(S \subseteq 2^S\), but \(S \not\approx 2^S\).
\end{question}
\begin{proof}
    To prove that for any set \(S\) we have \(S \subseteq 2^S\) but \(S \not\approx 2^S\), we proceed as follows:

    \textbf{1. Injection from \(S\) into \(2^S\):}  
    Define a function \(H: S \rightarrow {}^S 2\) such that \(H(x) \mapsto f_x\), where \(f_x: S \rightarrow \{0, 1\}\) is defined by:
    \[
    f_x(y) = 
    \begin{cases} 
    1 & \text{if } y = x \\
    0 & \text{otherwise}
    \end{cases}
    \]
    This function \(H\) is injective since distinct elements \(x, x' \in S\) will be mapped to distinct functions \(f_x\) and \(f_{x'}\).

    \textbf{2. Proving \(S \not\approx 2^S\):}  
    To show that \(S\) is not equinumerous to \(2^S\), we use a diagonal argument. Consider the function \(g: S \rightarrow \{0, 1\}\) defined by:
    \[
    g(x) = 1 - H(x)(x)
    \]
    We claim that the set \(B = \{g: g(x) = 1 - H(x)(x)\}\) is a subset of \({}^S 2\) but is not in the range of \(H\).

    \begin{proof} We prove by two parts: \\
        \(B \subseteq {}^S 2\): This is straightforward to show. Since \(H(x)(x) \in \{0, 1\}\), we have \(g(x) \in \{0, 1\}\) for all \(x \in S\). Therefore, \(g\) is a valid function from \(S\) to \(\{0, 1\}\), implying \(B \subseteq {}^S 2\). \\
        \(B\) is not in the range of \(H\): Suppose, for the sake of contradiction (BWOC), that there exists \(x \in S\) such that \(H(x) = g\). Then we have:
        \[
        H(x)(x) = 1 - H(x)(x)
        \]
        This implies:
        \[
        H(x)(x) = \frac{1}{2}
        \]
        which is a contradiction, since \(H(x)(x) \in \{0, 1\}\).

        Thus, we have shown that \(S \subseteq 2^S\) but \(S \not\approx 2^S\).
    \end{proof}

\end{proof}
\begin{question}
    \textbf{E 17}
    Give counterexamples to show that we cannot strengthen Theorem 6L by replacing "\(\leq\)" by "\(<\)" throughout.
\end{question}
Theorem 6L in Page 149:
\begin{leftbar}
    \textbf{Theorem 6L} Let \(\kappa\), \(\lambda\), and \(\mu\) be cardinal numbers.

    \begin{enumerate}
        \item[(a)] \(\kappa \leq \lambda \implies \kappa + \mu \leq \lambda + \mu\).
        \item[(b)] \(\kappa \leq \lambda \implies \kappa \cdot \mu \leq \lambda \cdot \mu\).
        \item[(c)] \(\kappa \leq \lambda \implies \kappa^\mu \leq \lambda^\mu\).
        \item[(d)] \(\kappa \leq \lambda \implies \mu^\kappa \leq \mu^\lambda\); if not both \(\kappa\) and \(\mu\) equal zero.
    \end{enumerate}
\end{leftbar}
\begin{proof}
    \textbf{A Concrete Example:}

    Consider the sets \(A = \{0, 1\}\) and \(B = \{0, 1, 2\}\).  
    We have \(|A| < |B|\), yet \(A \cup \omega \approx B \cup \omega\), which implies \(|A| + \aleph_0 = |B| + \aleph_0\).  
    The same holds for the other operations as well.

    \textbf{A More General and Abstract Example:}

    For any \(n, m \in \omega\) with \(n < m\), we have:
    \[
    n + \aleph_0 = m + \aleph_0, \quad n \times \aleph_0 = m \times \aleph_0,
    \]
    and similarly for the other cases.

\end{proof}



% \section{Inductive Definitions}

\subsection{Material Notes}


\subsection{Excercise Answers}
\begin{question}
    \textbf{E 6} Let $\kappa$ be a nonzero cardinal number. Show that there does not exist a set to which every set of cardinality $\kappa$ belongs.
\end{question}
\begin{proof}
    Proof by contradiction (BWOC): Assume that such a set exists. Let 
    \[ 
    A = \{ K \mid |K| = \kappa \} 
    \]
    be the set containing all sets with cardinality $\kappa$.

    Consider 
    \[
    \bigcup A,
    \]
    the union of all sets in $A$. By construction, $\bigcup A$ would be a set that contains every possible set of cardinality $\kappa$.

    However, this leads to a contradiction since such a set does not exist in the framework of standard set theory (e.g., due to limitations implied by Russell's paradox or cardinality constraints).

    Therefore, there does not exist a set $A$ such that it contains every set of cardinality $\kappa$.
\end{proof}

\begin{question}
    \textbf{E 7}
    Assume that $A$ is finite and $f: A \rightarrow A$. Show that $f$ is one-to-one if and only if $\text{ran } f = A$.
\end{question}
Given that $\text{ran } f = A$, this implies that $f$ is surjective.
Thus, the statement is equivalent to proving that: \textit{``Assume that $A$ is finite and $f: A \rightarrow A$. Show that $f$ is one-to-one if and only if $f$ is surjective.''}
\begin{proof}
Assume $A$ \text{ is finite.}
    \begin{itemize}
        \item (\(\Rightarrow\)) \textbf{(By way of contradiction):} Assume that $f$ is injective but not surjective. Then, the image of $f$ under $A$, denoted by $f[A]$, is a proper subset of $A$, meaning $f[A] \subsetneq A$. Thus, we have an injective function $f: A \rightarrow f[A]$. By construction, $f[A] = f[A]$, and therefore, $f$ is surjective onto $f[A]$. Hence, $f$ is a bijection from $A$ to $f[A]$, where $f[A] \subsetneq A$. According to Corollary 6D, this implies that $A$ is equinumerous to a proper subset of itself, which contradicts the assumption that $A$ is finite.

        \item (\(\Leftarrow\)) \textbf{(By way of contradiction):} Assume that $f$ is surjective but not injective. By the axiom of choice, there exists an injection, call it $g$, from $A$ to the preimage of $A$ under $f$, which is a proper subset of $A$. Such a function $g$ would then be a bijection from the preimage of $A$ to $A$. This again implies that $A$ is equinumerous to a proper subset of itself, leading to the conclusion that $A$ is infinite. This contradicts our original assumption that $A$ is finite.
    \end{itemize}
\end{proof}
\begin{question}
    \textbf{E 8}
    Prove that the union of two finite sets is finite (Corollary 6K), without any use of arithmetic.
\end{question}
\begin{proof}
    Let \(|A| \approx n\) and \(|B| \approx m\), where \(n, m \in \omega\). Assume that \(A\) and \(B\) are disjoint. By the definition of equinumerosity, there exist functions \(h: A \rightarrow n\) and \(g: B \rightarrow m\). We define a function \(f: A \cup B \rightarrow n + m\) (as defined earlier) such that

    \[
    f(x) = 
    \begin{cases} 
    h(x) & \text{if } x \in A \\
    n + g(x) & \text{if } x \in B
    \end{cases}
    \]

    This function is well-defined since \(A\) and \(B\) are disjoint. It is straightforward to verify that such an \(f\) is a bijection with an inverse \(f^{-1}: n + m \rightarrow A \cup B\) defined as follows:

    \[
    f^{-1}(k) = 
    \begin{cases} 
    h^{-1}(k) & \text{if } k \leq n \\
    g^{-1}(k - n) & \text{if } k > n
    \end{cases}
    \]

    If \(A\) and \(B\) are not disjoint, then let \(A \cap B = C \neq \emptyset\). Replace each element in \(C\) by elements not present in \(A \cup B\) to form a new set \(C'\). Let \(A' = (A - C) \cup C'\). We have \(|A'| = |A|\) and \(A \cup B \subseteq A' \cup B\), which forms a disjoint union. By Lemma 6F, there exists a \(k < n + m\) such that \(|A \cup B| = k\).

    Therefore, in both cases, \(A \cup B\) is finite.
\end{proof}
\begin{question}
    \textbf{E 9}
    Prove that the Cartesian product of two finite sets is finite (Corollary 6K), without any use of arithmetic.
\end{question}
\begin{proof}
    We will use induction to prove this statement:

    Let 
    \[
    S = \{ m \mid \forall n \in \omega, \forall A, B \text{ such that } |A| = m, |B| = n, |A \times B| = m \times n \},
    \]
    where \(\times\) is defined as the Cartesian product for natural numbers.

    \begin{enumerate}
        \item \textbf{Base Case:} \(0 \in S\). For any set \(B\), \(\emptyset \times B = \emptyset\). Thus, \(|\emptyset \times B| = 0 \times |B| = 0\) since \(B\) is finite and therefore \(|B| \in \omega\). Furthermore, \(1 \in S\) since for any singleton \(\{a\}\) and any set \(B\), there exists a bijection between \(\{a\} \times B\) and \(B\) given by the function \(f: \{a\} \times B \rightarrow B\) defined as \(f(a, x) = x\). The inverse function is \(g(x) = (a, x)\). Therefore, \(|\{a\} \times B| = 1 \times |B| = |B|\).
        
        \item \textbf{Inductive Step:} Assume \(k \in S\). We want to show that \(k^+ \in S\), where \(k^+\) denotes the successor of \(k\).
        
        To show \(k^+ \in S\), we need to prove that for all \(m \in \omega\) and for all sets \(A\) and \(B\) such that \(|A| = k^+\) and \(|B| = m\), we have \(|A \times B| = k^+ \times m\).
        
        Since we have already proven that \(0 \in S\), the induction starts with \(k \geq 1\), meaning \(k^+ \geq 2\).
        
        Let \(A\) be a set with \(|A| = k^+\). Since \(k^+ \geq 2\), there exists an element \(a \in A\). Let \(A' = A \setminus \{a\}\). Therefore, \(|A'| = k\), and by the induction hypothesis, \(|A' \times B| = k \times m\). Since \(\{a\}\) is a singleton, we have \(|\{a\} \times B| = 1 \times m = m\) by the induction hypothesis.
        
        Therefore,
        \[
        |A \times B| = |(A' \times B) \cup (\{a\} \times B)| = k \times m + m = k^+ \times m
        \]
        by the definition of multiplication for natural numbers.
    \end{enumerate}

    Hence, by induction, the Cartesian product of two finite sets is finite (since natural number is closed under multiplication).
\end{proof}
\begin{question}
    \textbf{E 10 to 12}
    For Excercises 10 to 12, proving all the Theorem 6I is suffices.
\end{question}
In Page 142
\begin{leftbar}
    \textbf{Theorem 6I} For any cardinal numbers \(\kappa\), \(\lambda\), and \(\mu\):

    \begin{enumerate}
        \item \(\kappa + \lambda = \lambda + \kappa\) and \(\kappa \cdot \lambda = \lambda \cdot \kappa\).
        \item \(\kappa + (\lambda + \mu) = (\kappa + \lambda) + \mu\) and \(\kappa \cdot (\lambda \cdot \mu) = (\kappa \cdot \lambda) \cdot \mu\).
        \item \(\kappa \cdot (\lambda + \mu) = \kappa \cdot \lambda + \kappa \cdot \mu\).
        \item \(\kappa^{\lambda + \mu} = \kappa^\lambda \cdot \kappa^\mu\).
        \item \((\kappa \cdot \lambda)^\mu = \kappa^\mu \cdot \lambda^\mu\).
        \item \((\kappa^\lambda)^\mu = \kappa^{\lambda \cdot \mu}\).
    \end{enumerate}

    \textbf{Proof} Take sets \(K\), \(L\), and \(M\) with \(\text{card } K = \kappa\), \(\text{card } L = \lambda\), and \(\text{card } M = \mu\); for convenience, choose them in such a way that any two are disjoint. Then each of the equations reduces to a corresponding statement about equinumerous sets. For example, \(\kappa \cdot \lambda = \text{card } (K \times L)\) and \(\lambda \cdot \kappa = \text{card } (L \times K)\); consequently, showing that \(\kappa \cdot \lambda = \lambda \cdot \kappa\) reduces to showing that \(K \times L \approx L \times K\). Listed in full, the statements to be verified are:

    \begin{enumerate}
        \item \(K \cup L \approx L \cup K\) and \(K \times L \approx L \times K\).
        \item \(K \cup (L \cup M) \approx (K \cup L) \cup M\) and \(K \times (L \times M) \approx (K \times L) \times M\).
        \item \(K \times (L \cup M) \approx (K \times L) \cup (K \times M)\).
        \item \({}^{(L \cup M)} K \approx {}^K L \times {}^k M\).
        \item \({}^M (K \times L) \approx {}^M K \times {}^M L\).
        \item \({}^{M} ({}^L K) \approx {}^{(L \times M)} K\).
    \end{enumerate}
\end{leftbar}
I will use the second part of 1 to 6 (on the set) to prove the theorem.
\begin{proof} For the simple ones I will simple list the fucntion without justifying that it is bijection.
    \begin{enumerate}
        \item \(K \cup L = L \cup K\) by algebra of set (Page 27). f: \(K \times L \rightarrow L \times K\) s.t. \(<k, l> \mapsto <l, k>\)
        \item \(K \cup (L \cup M) = (K \cup L) \cup M\) by algebra of set (Page 27). f: \(K \times (L \times M) \rightarrow (K \times L) \times M\) s.t. \(<k, <l, m>> \mapsto <<k, l>, m>\).
        \item \(K \times (L \cup M) = (K \times L) \cup (K \times M)\) by Excercise 2 Chapter3.
        \item \(H: {}^{(L \cup M)} K \rightarrow {}^K L \times {}^k M\) \(s.t. f \mapsto <f \upharpoonright {}_L, f \upharpoonright {}_{M - L}>\).
        \item \(H: {}^M (K \times L) \rightarrow {}^M K \times {}^M L\). Let \(f \in {}^M (K \times L) \), then \(\forall m \in M f: m \mapsto f(m) = <k, l> = <f_1(m), f_2(m)>\). Thus let \(H: f \mapsto <f_1, f_2>\).
        \item Proven in book. 
    \end{enumerate}
\end{proof}

\begin{question}
    \textbf{E 13}
    Show that a finite union of finite sets is finite. That is, show that if \(B\) is a finite set whose members are themselves finite sets, then \(\bigcup B\) is finite.
\end{question}
\begin{proof}
    In Chapter 6 E 3, we have proved that two finite set's union is finite. Do induction on natural number, we can prove that any finite union of finite sets is finite.
\end{proof}
\begin{question}
    \textbf{14}
    Define a \textit{permutation} of \(K\) to be any one-to-one function from \(K\) onto \(K\). We can then define the factorial operation on cardinal numbers by the equation
    \[
    \kappa! = \text{card} \{ f \mid f \text{ is a permutation of } K \},
    \]
    where \(K\) is any set of cardinality \(\kappa\). Show that \(\kappa!\) is well defined, i.e., the value of \(\kappa!\) is independent of just which set \(K\) is chosen.

\end{question}
\begin{proof}
    To show that \(\kappa!\) is well-defined, let \(\kappa\) be a cardinal number, and consider any two sets \(A\) and \(B\) such that \(|A| = |B| = \kappa\).

    Since \(A \approx A\), let \(f_A\) be a bijection \(f_A: A \rightarrow A\). Similarly, let \(f_B\) be a bijection \(f_B: B \rightarrow B\). Since \(A \approx B\), there exists a bijection \(f: A \rightarrow B\).

    We can establish a bijection between the set of permutations of \(A\) (denoted by \(\mathrm{perm}(A)\) or \(\mathrm{sym}(A)\)) and the set of permutations of \(B\) (denoted by \(\mathrm{perm}(B)\) or \(\mathrm{sym}(B)\)). This bijection can be constructed as follows: for any \(f_A \in \mathrm{perm}(A)\), we define a corresponding permutation of \(B\) by \(f_A' = f \circ f_B \circ f^{-1}\) (this is a conjugation by \(f\)).

    This construction is clearly a bijection between \(\mathrm{perm}(A)\) and \(\mathrm{perm}(B)\). Thus, the value of \(\kappa!\) is independent of the particular choice of the set \(K\) with cardinality \(\kappa\), proving that the factorial operation on cardinal numbers is well-defined.
\end{proof}
\begin{question}
    \textbf{E 15}
    Show that there is no set \(\mathcal{A}\) with the property that for every set there is some member of \(\mathcal{A}\) that dominates it.
\end{question}
\begin{proof}
    To prove that there is no such set \(\mathcal{A}\) with the stated property, we proceed by contradiction.

    \textbf{By way of contradiction (BWOC):} Assume that such a set \(\mathcal{A}\) exists. Let \(\mathcal{A}' = \bigcup \mathcal{A}\). By the definition of \(\mathcal{A}\), there exists a set \(K \in \mathcal{A}\) such that \(K\) dominates \(\mathcal{A}'\). Therefore, there is an injection from \(\mathcal{A}'\) to \(K\).

    By the definition of \(\mathcal{A}\), the power set \(2^K\) (the set of all subsets of \(K\)) is dominated by some element \(K' \in \mathcal{A}\). This means that there exists an injection from \(2^K\) to \(K'\). Since \(K' \in \mathcal{A}\) and \(\mathcal{A}' = \bigcup \mathcal{A}\), we have \(K' \subseteq \mathcal{A}'\). Therefore, the embedding of \(K'\) into \(\mathcal{A}'\) is a natural injection.

    Furthermore, we know that \(\mathcal{A}'\) injects into \(K\), and \(K\) injects into \(2^K\), since the power set of any set strictly dominates the set itself. Thus, we have an injection from \(2^K\) into \(\mathcal{A}'\). By the transitivity of injections (i.e., the property that injections are preserved under composition), this implies that \(2^K\) injects into \(K\), which contradicts the fact that the power set of a set always strictly dominates the set itself.

    Therefore, our initial assumption that such a set \(\mathcal{A}\) exists must be false.
\end{proof}

\begin{question}
    \textbf{E 16}
    Show that for any set \(S\) we have \(S \subseteq 2^S\), but \(S \not\approx 2^S\).
\end{question}
\begin{proof}
    To prove that for any set \(S\) we have \(S \subseteq 2^S\) but \(S \not\approx 2^S\), we proceed as follows:

    \textbf{1. Injection from \(S\) into \(2^S\):}  
    Define a function \(H: S \rightarrow {}^S 2\) such that \(H(x) \mapsto f_x\), where \(f_x: S \rightarrow \{0, 1\}\) is defined by:
    \[
    f_x(y) = 
    \begin{cases} 
    1 & \text{if } y = x \\
    0 & \text{otherwise}
    \end{cases}
    \]
    This function \(H\) is injective since distinct elements \(x, x' \in S\) will be mapped to distinct functions \(f_x\) and \(f_{x'}\).

    \textbf{2. Proving \(S \not\approx 2^S\):}  
    To show that \(S\) is not equinumerous to \(2^S\), we use a diagonal argument. Consider the function \(g: S \rightarrow \{0, 1\}\) defined by:
    \[
    g(x) = 1 - H(x)(x)
    \]
    We claim that the set \(B = \{g: g(x) = 1 - H(x)(x)\}\) is a subset of \({}^S 2\) but is not in the range of \(H\).

    \begin{proof} We prove by two parts: \\
        \(B \subseteq {}^S 2\): This is straightforward to show. Since \(H(x)(x) \in \{0, 1\}\), we have \(g(x) \in \{0, 1\}\) for all \(x \in S\). Therefore, \(g\) is a valid function from \(S\) to \(\{0, 1\}\), implying \(B \subseteq {}^S 2\). \\
        \(B\) is not in the range of \(H\): Suppose, for the sake of contradiction (BWOC), that there exists \(x \in S\) such that \(H(x) = g\). Then we have:
        \[
        H(x)(x) = 1 - H(x)(x)
        \]
        This implies:
        \[
        H(x)(x) = \frac{1}{2}
        \]
        which is a contradiction, since \(H(x)(x) \in \{0, 1\}\).

        Thus, we have shown that \(S \subseteq 2^S\) but \(S \not\approx 2^S\).
    \end{proof}

\end{proof}
\begin{question}
    \textbf{E 17}
    Give counterexamples to show that we cannot strengthen Theorem 6L by replacing "\(\leq\)" by "\(<\)" throughout.
\end{question}
Theorem 6L in Page 149:
\begin{leftbar}
    \textbf{Theorem 6L} Let \(\kappa\), \(\lambda\), and \(\mu\) be cardinal numbers.

    \begin{enumerate}
        \item[(a)] \(\kappa \leq \lambda \implies \kappa + \mu \leq \lambda + \mu\).
        \item[(b)] \(\kappa \leq \lambda \implies \kappa \cdot \mu \leq \lambda \cdot \mu\).
        \item[(c)] \(\kappa \leq \lambda \implies \kappa^\mu \leq \lambda^\mu\).
        \item[(d)] \(\kappa \leq \lambda \implies \mu^\kappa \leq \mu^\lambda\); if not both \(\kappa\) and \(\mu\) equal zero.
    \end{enumerate}
\end{leftbar}
\begin{proof}
    \textbf{A Concrete Example:}

    Consider the sets \(A = \{0, 1\}\) and \(B = \{0, 1, 2\}\).  
    We have \(|A| < |B|\), yet \(A \cup \omega \approx B \cup \omega\), which implies \(|A| + \aleph_0 = |B| + \aleph_0\).  
    The same holds for the other operations as well.

    \textbf{A More General and Abstract Example:}

    For any \(n, m \in \omega\) with \(n < m\), we have:
    \[
    n + \aleph_0 = m + \aleph_0, \quad n \times \aleph_0 = m \times \aleph_0,
    \]
    and similarly for the other cases.

\end{proof}



% \section{Chapter 1: Basic Concepts}

% \begin{question}
% What is the difference between operational and denotational semantics?
% \end{question}

% \begin{answer}
% Operational semantics provides a step-by-step procedure to evaluate a program, effectively describing how the program executes. Denotational semantics, on the other hand, maps each program to a mathematical object, often focusing on what the program computes rather than how.
% \end{answer}

% \begin{comment}
% This is an important distinction because operational semantics is more intuitive but may lack the mathematical rigor found in denotational semantics.
% \end{comment}

% \begin{question}
% Explain the use of abstract syntax in defining programming languages.
% \end{question}

% \begin{answer}
% Abstract syntax ignores details like parentheses and focuses on the structure of the expressions. For example, in arithmetic, \( e \rightarrow e + e \) is a rule that defines an expression without caring about the concrete appearance of parentheses.
% \end{answer}

% \begin{comment}
% I think abstract syntax trees (ASTs) provide an easy way to implement compilers by focusing on structure instead of syntax.
% \end{comment}

% \raynote{I need to explore more about the connection between operational semantics and compiler design.}

% \newpage
% \section{Chapter 2: Inductive Definitions}

% \begin{question}
% How can inductive definitions be used to define programming language semantics?
% \end{question}

% \begin{answer}
% Inductive definitions allow us to define objects step by step. In programming languages, they are used to define the structure of programs (syntax) and the way programs are evaluated (semantics). An example is defining valid expressions in a language using base cases and recursive steps.
% \end{answer}

% \begin{comment}
% Inductive definitions seem to be particularly useful when defining operational semantics.
% \end{comment}

\newpage
\section{Ray's Notes Summary}
\printraynotes

\end{document}
