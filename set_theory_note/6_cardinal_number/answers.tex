\subsection{Excercise Answers}
\begin{question}
    \textbf{E 6} Let $\kappa$ be a nonzero cardinal number. Show that there does not exist a set to which every set of cardinality $\kappa$ belongs.
\end{question}
\begin{proof}
    Proof by contradiction (BWOC): Assume that such a set exists. Let 
    \[ 
    A = \{ K \mid |K| = \kappa \} 
    \]
    be the set containing all sets with cardinality $\kappa$.

    Consider 
    \[
    \bigcup A,
    \]
    the union of all sets in $A$. By construction, $\bigcup A$ would be a set that contains every possible set of cardinality $\kappa$.

    However, this leads to a contradiction since such a set does not exist in the framework of standard set theory (e.g., due to limitations implied by Russell's paradox or cardinality constraints).

    Therefore, there does not exist a set $A$ such that it contains every set of cardinality $\kappa$.
\end{proof}

\begin{question}
    \textbf{E 7}
    Assume that $A$ is finite and $f: A \rightarrow A$. Show that $f$ is one-to-one if and only if $\text{ran } f = A$.
\end{question}
Given that $\text{ran } f = A$, this implies that $f$ is surjective.
Thus, the statement is equivalent to proving that: \textit{``Assume that $A$ is finite and $f: A \rightarrow A$. Show that $f$ is one-to-one if and only if $f$ is surjective.''}
\begin{proof}
Assume $A$ \text{ is finite.}
    \begin{itemize}
        \item (\(\Rightarrow\)) \textbf{(By way of contradiction):} Assume that $f$ is injective but not surjective. Then, the image of $f$ under $A$, denoted by $f[A]$, is a proper subset of $A$, meaning $f[A] \subsetneq A$. Thus, we have an injective function $f: A \rightarrow f[A]$. By construction, $f[A] = f[A]$, and therefore, $f$ is surjective onto $f[A]$. Hence, $f$ is a bijection from $A$ to $f[A]$, where $f[A] \subsetneq A$. According to Corollary 6D, this implies that $A$ is equinumerous to a proper subset of itself, which contradicts the assumption that $A$ is finite.

        \item (\(\Leftarrow\)) \textbf{(By way of contradiction):} Assume that $f$ is surjective but not injective. By the axiom of choice, there exists an injection, call it $g$, from $A$ to the preimage of $A$ under $f$, which is a proper subset of $A$. Such a function $g$ would then be a bijection from the preimage of $A$ to $A$. This again implies that $A$ is equinumerous to a proper subset of itself, leading to the conclusion that $A$ is infinite. This contradicts our original assumption that $A$ is finite.
    \end{itemize}
\end{proof}
\begin{question}
    \textbf{E 8}
    Prove that the union of two finite sets is finite (Corollary 6K), without any use of arithmetic.
\end{question}
\begin{proof}
    Let \(|A| \approx n\) and \(|B| \approx m\), where \(n, m \in \omega\). Assume that \(A\) and \(B\) are disjoint. By the definition of equinumerosity, there exist functions \(h: A \rightarrow n\) and \(g: B \rightarrow m\). We define a function \(f: A \cup B \rightarrow n + m\) (as defined earlier) such that

    \[
    f(x) = 
    \begin{cases} 
    h(x) & \text{if } x \in A \\
    n + g(x) & \text{if } x \in B
    \end{cases}
    \]

    This function is well-defined since \(A\) and \(B\) are disjoint. It is straightforward to verify that such an \(f\) is a bijection with an inverse \(f^{-1}: n + m \rightarrow A \cup B\) defined as follows:

    \[
    f^{-1}(k) = 
    \begin{cases} 
    h^{-1}(k) & \text{if } k \leq n \\
    g^{-1}(k - n) & \text{if } k > n
    \end{cases}
    \]

    If \(A\) and \(B\) are not disjoint, then let \(A \cap B = C \neq \emptyset\). Replace each element in \(C\) by elements not present in \(A \cup B\) to form a new set \(C'\). Let \(A' = (A - C) \cup C'\). We have \(|A'| = |A|\) and \(A \cup B \subseteq A' \cup B\), which forms a disjoint union. By Lemma 6F, there exists a \(k < n + m\) such that \(|A \cup B| = k\).

    Therefore, in both cases, \(A \cup B\) is finite.
\end{proof}
\begin{question}
    \textbf{E 9}
    Prove that the Cartesian product of two finite sets is finite (Corollary 6K), without any use of arithmetic.
\end{question}
\begin{proof}
    We will use induction to prove this statement:

    Let 
    \[
    S = \{ m \mid \forall n \in \omega, \forall A, B \text{ such that } |A| = m, |B| = n, |A \times B| = m \times n \},
    \]
    where \(\times\) is defined as the Cartesian product for natural numbers.

    \begin{enumerate}
        \item \textbf{Base Case:} \(0 \in S\). For any set \(B\), \(\emptyset \times B = \emptyset\). Thus, \(|\emptyset \times B| = 0 \times |B| = 0\) since \(B\) is finite and therefore \(|B| \in \omega\). Furthermore, \(1 \in S\) since for any singleton \(\{a\}\) and any set \(B\), there exists a bijection between \(\{a\} \times B\) and \(B\) given by the function \(f: \{a\} \times B \rightarrow B\) defined as \(f(a, x) = x\). The inverse function is \(g(x) = (a, x)\). Therefore, \(|\{a\} \times B| = 1 \times |B| = |B|\).
        
        \item \textbf{Inductive Step:} Assume \(k \in S\). We want to show that \(k^+ \in S\), where \(k^+\) denotes the successor of \(k\).
        
        To show \(k^+ \in S\), we need to prove that for all \(m \in \omega\) and for all sets \(A\) and \(B\) such that \(|A| = k^+\) and \(|B| = m\), we have \(|A \times B| = k^+ \times m\).
        
        Since we have already proven that \(0 \in S\), the induction starts with \(k \geq 1\), meaning \(k^+ \geq 2\).
        
        Let \(A\) be a set with \(|A| = k^+\). Since \(k^+ \geq 2\), there exists an element \(a \in A\). Let \(A' = A \setminus \{a\}\). Therefore, \(|A'| = k\), and by the induction hypothesis, \(|A' \times B| = k \times m\). Since \(\{a\}\) is a singleton, we have \(|\{a\} \times B| = 1 \times m = m\) by the induction hypothesis.
        
        Therefore,
        \[
        |A \times B| = |(A' \times B) \cup (\{a\} \times B)| = k \times m + m = k^+ \times m
        \]
        by the definition of multiplication for natural numbers.
    \end{enumerate}

    Hence, by induction, the Cartesian product of two finite sets is finite (since natural number is closed under multiplication).
\end{proof}
\begin{question}
    \textbf{E 10 to 12}
    For Excercises 10 to 12, proving all the Theorem 6I is suffices.
\end{question}
In Page 142
\begin{leftbar}
    \textbf{Theorem 6I} For any cardinal numbers \(\kappa\), \(\lambda\), and \(\mu\):

    \begin{enumerate}
        \item \(\kappa + \lambda = \lambda + \kappa\) and \(\kappa \cdot \lambda = \lambda \cdot \kappa\).
        \item \(\kappa + (\lambda + \mu) = (\kappa + \lambda) + \mu\) and \(\kappa \cdot (\lambda \cdot \mu) = (\kappa \cdot \lambda) \cdot \mu\).
        \item \(\kappa \cdot (\lambda + \mu) = \kappa \cdot \lambda + \kappa \cdot \mu\).
        \item \(\kappa^{\lambda + \mu} = \kappa^\lambda \cdot \kappa^\mu\).
        \item \((\kappa \cdot \lambda)^\mu = \kappa^\mu \cdot \lambda^\mu\).
        \item \((\kappa^\lambda)^\mu = \kappa^{\lambda \cdot \mu}\).
    \end{enumerate}

    \textbf{Proof} Take sets \(K\), \(L\), and \(M\) with \(\text{card } K = \kappa\), \(\text{card } L = \lambda\), and \(\text{card } M = \mu\); for convenience, choose them in such a way that any two are disjoint. Then each of the equations reduces to a corresponding statement about equinumerous sets. For example, \(\kappa \cdot \lambda = \text{card } (K \times L)\) and \(\lambda \cdot \kappa = \text{card } (L \times K)\); consequently, showing that \(\kappa \cdot \lambda = \lambda \cdot \kappa\) reduces to showing that \(K \times L \approx L \times K\). Listed in full, the statements to be verified are:

    \begin{enumerate}
        \item \(K \cup L \approx L \cup K\) and \(K \times L \approx L \times K\).
        \item \(K \cup (L \cup M) \approx (K \cup L) \cup M\) and \(K \times (L \times M) \approx (K \times L) \times M\).
        \item \(K \times (L \cup M) \approx (K \times L) \cup (K \times M)\).
        \item \({}^{(L \cup M)} K \approx {}^K L \times {}^k M\).
        \item \({}^M (K \times L) \approx {}^M K \times {}^M L\).
        \item \({}^{M} ({}^L K) \approx {}^{(L \times M)} K\).
    \end{enumerate}
\end{leftbar}
I will use the second part of 1 to 6 (on the set) to prove the theorem.
\begin{proof} For the simple ones I will simple list the fucntion without justifying that it is bijection.
    \begin{enumerate}
        \item \(K \cup L = L \cup K\) by algebra of set (Page 27). f: \(K \times L \rightarrow L \times K\) s.t. \(<k, l> \mapsto <l, k>\)
        \item \(K \cup (L \cup M) = (K \cup L) \cup M\) by algebra of set (Page 27). f: \(K \times (L \times M) \rightarrow (K \times L) \times M\) s.t. \(<k, <l, m>> \mapsto <<k, l>, m>\).
        \item \(K \times (L \cup M) = (K \times L) \cup (K \times M)\) by Excercise 2 Chapter3.
        \item \(H: {}^{(L \cup M)} K \rightarrow {}^K L \times {}^k M\) \(s.t. f \mapsto <f \upharpoonright {}_L, f \upharpoonright {}_{M - L}>\).
        \item \(H: {}^M (K \times L) \rightarrow {}^M K \times {}^M L\). Let \(f \in {}^M (K \times L) \), then \(\forall m \in M f: m \mapsto f(m) = <k, l> = <f_1(m), f_2(m)>\). Thus let \(H: f \mapsto <f_1, f_2>\).
        \item Proven in book. 
    \end{enumerate}
\end{proof}

\begin{question}
    \textbf{E 13}
    Show that a finite union of finite sets is finite. That is, show that if \(B\) is a finite set whose members are themselves finite sets, then \(\bigcup B\) is finite.
\end{question}
\begin{proof}
    In Chapter 6 E 3, we have proved that two finite set's union is finite. Do induction on natural number, we can prove that any finite union of finite sets is finite.
\end{proof}
\begin{question}
    \textbf{14}
    Define a \textit{permutation} of \(K\) to be any one-to-one function from \(K\) onto \(K\). We can then define the factorial operation on cardinal numbers by the equation
    \[
    \kappa! = \text{card} \{ f \mid f \text{ is a permutation of } K \},
    \]
    where \(K\) is any set of cardinality \(\kappa\). Show that \(\kappa!\) is well defined, i.e., the value of \(\kappa!\) is independent of just which set \(K\) is chosen.

\end{question}
\begin{proof}
    To show that \(\kappa!\) is well-defined, let \(\kappa\) be a cardinal number, and consider any two sets \(A\) and \(B\) such that \(|A| = |B| = \kappa\).

    Since \(A \approx A\), let \(f_A\) be a bijection \(f_A: A \rightarrow A\). Similarly, let \(f_B\) be a bijection \(f_B: B \rightarrow B\). Since \(A \approx B\), there exists a bijection \(f: A \rightarrow B\).

    We can establish a bijection between the set of permutations of \(A\) (denoted by \(\mathrm{perm}(A)\) or \(\mathrm{sym}(A)\)) and the set of permutations of \(B\) (denoted by \(\mathrm{perm}(B)\) or \(\mathrm{sym}(B)\)). This bijection can be constructed as follows: for any \(f_A \in \mathrm{perm}(A)\), we define a corresponding permutation of \(B\) by \(f_A' = f \circ f_B \circ f^{-1}\) (this is a conjugation by \(f\)).

    This construction is clearly a bijection between \(\mathrm{perm}(A)\) and \(\mathrm{perm}(B)\). Thus, the value of \(\kappa!\) is independent of the particular choice of the set \(K\) with cardinality \(\kappa\), proving that the factorial operation on cardinal numbers is well-defined.
\end{proof}
\begin{question}
    \textbf{E 15}
    Show that there is no set \(\mathcal{A}\) with the property that for every set there is some member of \(\mathcal{A}\) that dominates it.
\end{question}
\begin{proof}
    To prove that there is no such set \(\mathcal{A}\) with the stated property, we proceed by contradiction.

    \textbf{By way of contradiction (BWOC):} Assume that such a set \(\mathcal{A}\) exists. Let \(\mathcal{A}' = \bigcup \mathcal{A}\). By the definition of \(\mathcal{A}\), there exists a set \(K \in \mathcal{A}\) such that \(K\) dominates \(\mathcal{A}'\). Therefore, there is an injection from \(\mathcal{A}'\) to \(K\).

    By the definition of \(\mathcal{A}\), the power set \(2^K\) (the set of all subsets of \(K\)) is dominated by some element \(K' \in \mathcal{A}\). This means that there exists an injection from \(2^K\) to \(K'\). Since \(K' \in \mathcal{A}\) and \(\mathcal{A}' = \bigcup \mathcal{A}\), we have \(K' \subseteq \mathcal{A}'\). Therefore, the embedding of \(K'\) into \(\mathcal{A}'\) is a natural injection.

    Furthermore, we know that \(\mathcal{A}'\) injects into \(K\), and \(K\) injects into \(2^K\), since the power set of any set strictly dominates the set itself. Thus, we have an injection from \(2^K\) into \(\mathcal{A}'\). By the transitivity of injections (i.e., the property that injections are preserved under composition), this implies that \(2^K\) injects into \(K\), which contradicts the fact that the power set of a set always strictly dominates the set itself.

    Therefore, our initial assumption that such a set \(\mathcal{A}\) exists must be false.
\end{proof}

\begin{question}
    \textbf{E 16}
    Show that for any set \(S\) we have \(S \subseteq 2^S\), but \(S \not\approx 2^S\).
\end{question}
\begin{proof}
    To prove that for any set \(S\) we have \(S \subseteq 2^S\) but \(S \not\approx 2^S\), we proceed as follows:

    \textbf{1. Injection from \(S\) into \(2^S\):}  
    Define a function \(H: S \rightarrow {}^S 2\) such that \(H(x) \mapsto f_x\), where \(f_x: S \rightarrow \{0, 1\}\) is defined by:
    \[
    f_x(y) = 
    \begin{cases} 
    1 & \text{if } y = x \\
    0 & \text{otherwise}
    \end{cases}
    \]
    This function \(H\) is injective since distinct elements \(x, x' \in S\) will be mapped to distinct functions \(f_x\) and \(f_{x'}\).

    \textbf{2. Proving \(S \not\approx 2^S\):}  
    To show that \(S\) is not equinumerous to \(2^S\), we use a diagonal argument. Consider the function \(g: S \rightarrow \{0, 1\}\) defined by:
    \[
    g(x) = 1 - H(x)(x)
    \]
    We claim that the set \(B = \{g: g(x) = 1 - H(x)(x)\}\) is a subset of \({}^S 2\) but is not in the range of \(H\).

    \begin{proof} We prove by two parts: \\
        \(B \subseteq {}^S 2\): This is straightforward to show. Since \(H(x)(x) \in \{0, 1\}\), we have \(g(x) \in \{0, 1\}\) for all \(x \in S\). Therefore, \(g\) is a valid function from \(S\) to \(\{0, 1\}\), implying \(B \subseteq {}^S 2\). \\
        \(B\) is not in the range of \(H\): Suppose, for the sake of contradiction (BWOC), that there exists \(x \in S\) such that \(H(x) = g\). Then we have:
        \[
        H(x)(x) = 1 - H(x)(x)
        \]
        This implies:
        \[
        H(x)(x) = \frac{1}{2}
        \]
        which is a contradiction, since \(H(x)(x) \in \{0, 1\}\).

        Thus, we have shown that \(S \subseteq 2^S\) but \(S \not\approx 2^S\).
    \end{proof}

\end{proof}
\begin{question}
    \textbf{E 17}
    Give counterexamples to show that we cannot strengthen Theorem 6L by replacing "\(\leq\)" by "\(<\)" throughout.
\end{question}
Theorem 6L in Page 149:
\begin{leftbar}
    \textbf{Theorem 6L} Let \(\kappa\), \(\lambda\), and \(\mu\) be cardinal numbers.

    \begin{enumerate}
        \item[(a)] \(\kappa \leq \lambda \implies \kappa + \mu \leq \lambda + \mu\).
        \item[(b)] \(\kappa \leq \lambda \implies \kappa \cdot \mu \leq \lambda \cdot \mu\).
        \item[(c)] \(\kappa \leq \lambda \implies \kappa^\mu \leq \lambda^\mu\).
        \item[(d)] \(\kappa \leq \lambda \implies \mu^\kappa \leq \mu^\lambda\); if not both \(\kappa\) and \(\mu\) equal zero.
    \end{enumerate}
\end{leftbar}
\begin{proof}
    \textbf{A Concrete Example:}

    Consider the sets \(A = \{0, 1\}\) and \(B = \{0, 1, 2\}\).  
    We have \(|A| < |B|\), yet \(A \cup \omega \approx B \cup \omega\), which implies \(|A| + \aleph_0 = |B| + \aleph_0\).  
    The same holds for the other operations as well.

    \textbf{A More General and Abstract Example:}

    For any \(n, m \in \omega\) with \(n < m\), we have:
    \[
    n + \aleph_0 = m + \aleph_0, \quad n \times \aleph_0 = m \times \aleph_0,
    \]
    and similarly for the other cases.

\end{proof}


